%%%%%%%%%%%%%%%%%%%%%%%%%%%%%%%%%%%%%%%%%%
%Copyright (C) 2018-2019 YuZJ Lab.
%使用CC-BY-NC-SA授权。一份完整版本的许可证已位于附录。这个版本原始作者YuZJ,
%邮箱\theafamily@126.com(最后连接于2019年06月20日17:32:17)。
%%%%%%%%%%%%%%%%%%%%%%%%%%%%%%%%%%%%%%%%%%
\chapter{跋}
在创作这本书的时候,我深深地感到了我语言的苍白、英语水平低下,对计算机知识的了解也仅仅浮于表面。我不得不对在计算机方面雄心勃勃的电教委员说,学好英语是第一要务。\par
“技”海无涯,这本书将永远“不”完全,并会不断完善下去。恳请广大读者不吝赐教。
\chapter{强大的生产力所需的网络架构:\\以Ubuntu Server 19.04为操作系统搭建服务器}
这一部分是针对电教员的。你需要安装一台具有固定IP(或者你可以让你的用户在每一次需要使用时打你的电话问一下当前服务器IP地址)的Ubuntu Server 19.04(或者其它类似的操作系统)。
\section{防火墙与固定IP}
若需要设置固定IP,请使用文本编辑器编辑“/etc/netplan/”目录下的一个扩展名为“yaml”的文件。你将得到:
\begin{verbatim}
1 # This file describes the network interfaces available on your system
2 # For more information, see netplan(5).
3 network:
4   version: 2
5   renderer: networkd
6   ethernets:
7     ##这里应该显示网卡名称。
8       addresses: [192.168.0.101/24]##IP地址及网络号长度(对应子网掩码255.255.255.0)
9       gateway4: 192.168.0.1##默认网关
10       nameservers:##DNS服务器
11         addresses: [8.8.8.8]
12
##前面的数字是Vim显示的行号。
\end{verbatim}\par
我们现在使用一款名为“ufw”的防火墙软件(从软件源里安装它)。ufw能够进行一些简单的网络设置。注意更改设置需要使用root权限。常用的命令如(具体请参照man手册):
\begin{verbatim}
ufw enable##开启ufw
ufw disable##关闭ufw
ufw allow in to 192.168.0.102##允许来自于192.168.0.102的进入。
##这个命令语法是“ufw [规则]”,上例中“allow in to 192.168.0.102”属于[规则]。
##这个规则语法是
[allow/deny/reject] proto [协议名] [in/out] on [网卡名] from [IP地址] to [IP地址]
##其它常见规则语法:
[allow/deny/reject][端口号]/[协议名,如TCP]
[allow/deny/reject][协议名,如SMTP]
##deny为禁止进入(丢弃传入的数据包),reject为拒绝进入(返回一个错误信息)。
ufw defaut [allow/deny/reject]##选择默认政策。
ufw logging [on/off]##是否使用日志。
ufw delete [规则]##删除规则。
\end{verbatim}
\section{使用VSFTPD配置FTP服务器}
VSFTPD(非常安全FTP守护进程)是一个在GNU/Linux平台被广泛使用的ftp服务器。下面以Ubuntu Server 19.04为例教你如何使用VSFTPD。
\subsection{下载安装}
这个太显然了吧。
\begin{verbatim}
sgcomputers@server:~$sudo apt install vsftpd
Reading packagelists... Done
Building dependency tree
##以下省略一大堆内容。
Processing triggers for systemd (240-6ubuntu5.1) …
\end{verbatim}
\subsection{配置VSFTPD}
此时我建议你先使用“\verb|su|”命令切换到root账户。这能节省时间。目前我的VSFTPD最高版本为3.0.3,配置文件是“/etc/vsftpd.conf”。使用Vim或其它你喜欢的编辑器打开它,取消注释(删除“\verb|#|”)以下内容:
\begin{verbatim}
write_enable=YES
##这样做使用户具有上传信息的权限。
anonymous_enable=YES
##这样做允许匿名登录。
chroot_list_enable=YES
##这样做允许使用内建用户的用户名和密码登录FTP(不安全,但操作方便)
chroot_list_file=/etc/vsftpd.chroot_list
##这张表列出允许使用上述方法登陆的内建用户名。
local_root=/var/ftp
##修改FTP根目录。否则将会作为使用上述方法登陆的内建用户的“家”目录。
\end{verbatim}\par
现在以“ftplink”为例添加使用上述方法登陆的内建用户。
\begin{verbatim}
root@server:/etc#useradd ftplink
##新建用户
root@server:/etc#passwd ftplink
##修改密码
Newpassword:
Retypenew password:
passwd:password updated successfully
root@server:/#mkdir /home/ftplink
\end{verbatim}\par
别忘了到“/etc/vsftpd.chroot\_list”添加ftplink。
\subsection{配置文件系统}
\begin{verbatim}
root@server:/etc#mkdir /var/ftp
##新建FTP根目录
root@server:/etc#cd /var/ftp
root@server:/var/ftp#chown root.root /var/ftp
##请注意,VSFTPD要求FTP根目录不可写。
root@server:/var/ftp#chmod a-w /var/ftp
root@server:/var/ftp#mkdir main
root@server:/var/ftp#chown ftplink.ftplink main
root@server:/var/ftp#chmod u+w+r+x /var/ftp/main
##建立可写目录main。
root@server:/j# chown root.root/etc/vsftpd*
root@server:/# chmod a-r-w-x /etc/vsftpd*
##保证VSFTPD配置文件不能被其它用户读取。
\end{verbatim}\par
修改“/home/ftplink/.bashrc”文件,删除所有内容并添加
\begin{verbatim}
logout
exit
\end{verbatim}\par
来确保没人可以通过Telnet远程登录这个用户。
\subsection{调试}
\begin{verbatim}
root@server:/#service vsftpd start
\end{verbatim}\par
开始服务,现在请使用你喜欢的FTP客户端试验上传、下载、删除文件等操作吧。如果你需要配置多个用户,只需要重复这些操作就行了。注意你需要修改每个用户的目录权限并允许特定用户通过防火墙。
\section{在Windows下配置FTP服务器}
你不需要使用ServU,有一个自由免费的软件可以替代它——Filezilla Server。你不需要使用Windows Server(它的IIS已经可以搭建优秀的服务器了),一台具有固定IP的Winsows10即可!到它的官网上面下载安装吧!
\begin{center}
	\includegraphics[scale=0.6]{pic/fzs}
\end{center} \par
从上图中可以看到整个服务器是完全图形界面的。现在请跟我做:\par
1.打开防火墙。找到“控制面板”-“系统和安全”-“Windows Defender防火墙”-“允许应用通过Windows防火墙”,单击“更改设置”-“允许其他应用”,添加Filezilla Server(它应为于C:\textbackslash Program Files (x86)\textbackslash FileZilla Server\textbackslash FileZilla Server.exe)。“确定”保存。\par
2.现在开始配置服务器。双击图标进入服务器,对弹出框单击“Connect”(就是要连接本地服务器)。现在进入“Edit”-“Users”添加用户。在“General”页中单击“Add”添加用户和密码,在“Shared folders”页针对每一个用户添加目录和对应的权限即可。这样一个简单地FTP服务器就设置好了。
\section{使用Telnetd配置远程登录服务器}
这太简单了,安装“telnetd”(CentOS是“telnet-server”)软件包并设置防火墙即可。
\section{使用SSHD与Git配置Git服务器}
现在将使用sshd与git配置git版本控制服务器。首先使用软件源安装这两个应用,设置好防火墙并使用“\verb|sudo service sshd start|”开启sshd服务。大部分命令与上一节创建FTP服务器类似,这里不再赘述。
\subsection{初始化}
先使用“\verb|useradd git|”新建用户“git”并修改密码,新建“/home/git/.ssh”目录并在里面创建“ authorized\_keys”文件。为了允许授权的用户访问服务器,请各位授权的用户运行“ssh-keygen”命令并将产生的“id\_rsa.pub”公钥文件发送到电教员手中(比如说,使用FTP),电教员将它们中的内容添加到“ authorized\_keys”文件末尾,一条公钥一行。\par
为了防止入侵,请修改“/etc/passwd”文件的“git”用户对应的一行末尾“/bin/bash”为“/usr/bin/git-shell”,使他大致呈现“git:x:xxxx:xxxx:xxx:/home/git:/usr/bin/git-shell”(某些内容用x表示)。
\subsection{创建仓库}
到“/opt/git”目录,使用“\verb|git init --bare --share [仓库名].git|”新建仓库并修改文件所有者为“git”并可读可写可执行,此时远程的用户就可以通过“\verb| git clone git@[服务器地址]:/opt/git/[仓库名].git|”来克隆这个仓库。未经授权的用户可以输入git用户的密码来访问。
\section{从此告别U盘}
我相信所有的电教员一定很反感U盘在教育教学中的使用:U盘不仅成为了病毒的“特快车道”,还有保密度低、容易遗失的不足。通过完善的网络配置即可除去U盘在教育教学中的使用。\par
首先,请使用硬件(这个太简单了吧,把线剪掉就好了)或软件(推荐这一种。可以使用火绒的相关功能)方法禁用学生机的USB接口。之后请搭建一个高速FTP服务器,确保每台教师机和学生机都能登录服务器。这样需要什么教学资源只要下载即可。或者也可以搭建无盘服务器,将教学资源放在服务器上。
\chapter{GNU宣言}
GNU宣言\footnote{我将这份历史文件添加进本技术指南的目的,是为了让程序员在商业化如潮水般推进时,能够关注一些其它理念。该份文件仅代表自由软件运动捍卫者的观点,并不能代表YuZJ Lab的观点或立场。\cite{gnum},被引用时文章内所有链接均有效。注意,有些脚注是由GNU CTT加的。}(如下所示)由\href{http://www.stallman.org/}{Richard Stallman}在1985年撰写,用来请求大家支持GNU操作系统的开发。其部分文本摘自1983年撰写的初始声明。直到1987年,因为开发的原因它时时小有更改;那时起,看起来最好是保持它不再改变。\par
时过境迁,我们认识到使用不同的措辞可以避免一些常见的误解。从1993年起,我们添加了脚注来澄清这些问题。\par
如果你想安装GNU/Linux系统,我们建议你使用\href{http://www.gnu.org/distros}{100\%自由的GNU/Linux发行版}之一。如果你想做出贡献,请参看\url{http://www.gnu.org/help/help.html}。\par
GNU工程是自由软件运动的一部分,该运动旨在\href{http://www.gnu.org/philosophy/free-sw.html}{捍卫软件用户的自由}。把GNU和“开源”一词联系在一起是错误的—该词汇是1998年由一些不赞同自由软件运动之道德价值的人士发明的。他们使用该词汇来推动同一领域的\href{http://www.gnu.org/philosophy/open-source-misses-the-point.html}{非道德方案}。\par
\section{GNU为何?GNU并非UNIX!}
GNU,代表的是Gnu's NotUnix(GNU并非UNIX),是我正在编写的一个完全兼容Unix的软件系统,这样我就可以把它自由地交给想要使用它的人。\footnote{此处用词不当。其初衷是人们不必为使用GNU系统而支付许可费。但是用词却没有清楚地说明此事,而人们经常理解为这是说GNU的拷贝总是免费或廉价地发行。这不是本意;后来,宣言指出公司提供有偿发行服务的可能性。之后,我也了解到认真区别自由中的“free(自由)”和价格中的“free(免费)”。自由软件是用户有自由修改和发布的软件。有些用户可能得到免费拷贝,而有些用户付费得到拷贝—如果这些资金帮助到软件的改善,善莫大焉。重要的一点是拥有拷贝的用户有自由和其他人一起使用自由软件。}还有几个志愿者在帮助我。我们非常需要大家在时间、金钱、程序和设备方面的贡献。\par
目前,我们有一个可以用lisp编写编辑命令的Emacs文本编辑器、一个源代码级别的调试器、一个兼容yacc的分析器生成工具、一个链接器和大约35个应用程序。shell(命令解释器)也接近完成。一个新的可移植的优化C编译器已经可以自我编译,可能会在年内发布。现有一个初始的内核,不过还需要增加很多功能才可以模拟Unix。当内核和编译器完成后,我们就有可能发布一个适合开发程序的GNU系统。我们会使用Tex作为文本排版工具,不过nroff还需要一些工作。我们还会使用自由的、可移植的XWindow系统。此后,我们还会加入一个可移植的CommonLisp、一个Empire游戏、一个电子表格和数百个应用以及在线文档。最终,我们希望提供Unix系统常规带有的一切有用之物,以及更多。\par
GNU将能够运行Unix的程序,但是它不完全和Unix一样。我们会根据我们在其他操作系统上的感受做出所有合理的改进。特别地,我们计划使用更长的文件名、文件版本号、防崩溃的文件系统、也许带有文件名填充、终端无关的显示支持、最后可能有一个基于Lisp的窗口系统,此时Lisp程序和普通Unix程序可以共享一个屏幕。C和Lisp都将作为系统编程语言。我们会支持UUCP、MITChaosnet和Internet等通信协议。\par
GNU最初的目标是68000/16000之类的带虚拟内存的机器,因为它们是最容易跑起来的机器。让GNU在更小的机器上运行的额外努力就留给那些需要使用这些机器的人吧。\par
为了避免可怕的混淆,请在指示本工程时,发出“GNU”中g的音。
\section{为什么我必须编写GNU}
我认可的黄金法则是如果我喜欢一个程序,我就必须把它分享给喜欢它的人。软件销售商通过让每个用户保证不和其他人分享来分化用户并控制他们。我拒绝以这种方式打破和其他用户组成的统一体。我的良知让我无法签署这样的保密协议或软件许可证协议。几年来,我在人工智能实验室都在反抗这种趋势以及其他冷漠,但是最终他们还是走得太远了:我无法再呆在一个为我做违背我意愿之事的机构。\par
为了能够继续不失颜面地使用计算机,我决定把一些必要的自由软件集合在一起,这样我就能够继续下去而不需要任何非自由软件。我从人工智能实验室辞了职,这样就可以在我发布GNU时避免和MIT产生法律纠葛。\footnote{2.“赠送”是另一个不妥的表达,它再次说明我那时还没有清楚地分开价格和自由的问题。我们现在建议在谈论自由软件时避免这一表达。请参看\href{http://www.gnu.org/philosophy/words-to-avoid.html\#GiveAwaySoftware}{“不清楚的词汇和短语”}了解更多解释}\par
\section{为什么GNU将会兼容Unix}
Unix并不是我理想中的系统,但是它还不算太差。Unix的主要功能看来是好的,而我认为我可以在不破坏这些好功能的情况下填补Unix缺少的东西。而且和Unix兼容可以让许多人能够方便地接纳它。\par
\section{如何获取GNU}
GNU不属于公共领域。GNU允许任何人修改和再发布,但是任何发布者都不能限制它的继续发布。就是说,它不允许专有性的修改。我想让GNU的所有版本都保持自由。\par
\section{为什么许多程序员想要提供帮助}
我发现许多程序员看到GNU很兴奋并想要提供帮助。\par
许多程序员对系统软件的商业化并不高兴。这可能使他们赚到更多的钱,不过这一般要求他们和其他程序员之间是对立关系,而不是伙伴关系。程序员之间的友谊的基本方式是分享程序;而现在典型的市场活动基本上是禁止程序员互相成为朋友。软件买家必须在友谊和守法之间抉择。自然地,许多人认为友谊更重要。但是许多守法的人通常会感到选哪个都不自在。他们变得愤世嫉俗并且认为编程只是一个挣钱的手段。\par
开发GNU和使用GNU而不是专有软件,我们就能够变得友善并守法。另外,GNU成为一个激励和团结其他人加入分享行列的榜样和旗帜。这给予我们一种和谐的感觉,它是使用非自由软件不可能有的。就和我讨论过的程序员来说,大约一半人认为这是一个重要的幸福感,而它是金钱无法替代的。\par
\section{你该如何做出贡献}
(现今,软件帮助任务请看\href{http://fsf.org/campaigns/priority-projects}{高优先级项目列表}和\href{http://savannah.gnu.org/people/?type_id=1}{GNU帮助需求列表},这是GNU软件包的一般任务列表。其他帮助,请看\href{http://www.gnu.org/help/help.html}{帮助GNU操作系统的指南}。)\par
我请求计算机制造商捐助机器和金钱。我请求个人捐助程序和作品。\par
如果你捐助机器,你可以期待的结果就是GNU将会早一天在该机器上运行。捐助的机器应该是完备的、可用的系统,它应该适用于居家的环境,并无需复杂的冷却或供电系统。\par
我已经找到相当多的程序员,他们热切地想要为GNU贡献闲暇时的工作。就大多数项目而言,这些工作很难协调;这些独立完成的部分凑在一起会不工作。但是就替代Unix的特定任务而言,没有这个问题。一个完整的Unix系统包含数百个应用程序,每个都有独立的文档。大多数的接口规格都由Unix兼容性所限定。如果每个贡献者能够编写一个单一的兼容性Unix应用,并使之在原始的Unix系统中正常工作,那么这些应用放在一起就会正常工作。即使出现一些意外的墨菲问题\footnote{1.Murphy,墨菲效应。是指事情如果有变坏的可能,不管这种可能性有多小,它总会发生。},联合这些部件也是可以完成的任务。(内核将需要更密切的沟通,它将会由一个小的、紧凑的小组来进行。)\par
如果我得到金钱上的捐助,我也许能够雇佣一些全职或兼职的人。薪水按照程序员的标准来看的话不高,但是我要找的人要和看重金钱一样看中社区精神的建设。我把这当作一种方法,它让一些人能够全身心地为GNU工作而不用寻求其他谋生的手段。
\section{为什么所有计算机用户都会受益}
一旦GNU完成,任何人都能够自由地得到一个好用的系统,正如得到空气一样。\footnote{这是又一个我没有认真区别“free”一词的两种意思的地方。该陈述并没有错—你是可以免费获得GNU软件,从朋友那里或从网上下载。但是它在提倡错误的理念。}\par
其意义远远超出了只是为每个人省去一份Unix许可证费用。这意味着避免了大量重复的系统编程工作造成的浪费。这些努力就可以用于推进技术的进步。\par
完整的系统资源将向每个人开放。其结果是,如果有用户需要更改系统,他总可以自由地自己修改或雇用其他程序员或公司来改。用户就用不再祈求拥有源代码的那一家公司或那一个程序员来帮他修改,没有人再处于独断的地位。\par
通过鼓励学生学习和改进系统代码,学校能够提供多得多的教育环境。哈佛大学的计算机实验室曾有一个政策:如果程序的源代码不能公开显示在屏幕上,那么就不能安装该程序,这就是坚持拒绝安装某些程序。我受此启发良多。\par
最后,考虑谁是系统软件的所有者以及谁应该做或不做什么的开销也被化解了。\par
筹划人们为一个程序付费,包括许可证费用,因为要通过麻烦的机制来搞清楚一个人应该为该程序支付多少费用,总是会导致大量的社会成本。而且只有管制的国家才能强制每个人都遵守付费制度。举例来说,空间站的空气要花大量成本来制造:为每次呼吸的容量计费是公平的,但是时时都带着测量面具即使是对负担得起呼吸费用的人也是无法忍受的事。加之随处可见的、监控人们是否脱掉面具的摄像头也令人无法容忍。所以,支持空气工厂的最好办法还是只收人头税并摆脱掉面具。\par
复制全部或部分程序对程序员来说和呼吸一样自然,一样有生产力。它也应该一样自由。\par
\section{一些容易驳斥的、反对GNU目标的观点}
“如果免费,就没有人会用了,因为用户没有可靠的技术支持。”\par
“你必须对程序收钱才能提供技术支持。”\par
如果人们宁愿免费获得没有服务的GNU,而不是付费给GNU获得服务,那么为免费GNU提供技术服务的公司应该是有利可图的。\footnote{现在就有几个这样的公司。}\par
我们必须区别对待真正的编程和仅仅是手把手服务这两种形式的技术支持。前者是你不能依赖一个软件供应商来解决的。如果你的问题没有被足够多的人共同体会,那么供应商会告诉你:快走开。\par
如果你的业务需要依赖于技术支持,那么唯一的办法是拥有所有必要的源代码和工具。然后,你就可以雇佣任何有能力的人为你解决问题;你就不必祈求某个特定的人。对Unix,源代码的价格使大多数人都不会考虑。对GNU,这就简单了。还会有找不到能人的时候,但这个问题不是发行策划的问题。GNU并没有解决世界上所有的问题,只是其中一些问题。\par
同时,对计算机知之甚少的用户需要手把手服务:为他们做些很容易但他们真的不知道怎么做的事。\par
这些服务可以由那些只销售手把手服务和修复服务的公司提供。如果用户愿意花钱买带服务的产品,那么他们也应该会为免费的产品购买服务。服务公司竞争的是质量和价格;用户不会绑定在某个服务商上。同时,像我们这样的不需要服务的人可以不用购买服务来使用程序。\par
“不打广告,不可能有很多人知道,所以你必须对程序收费才能够支付广告费。”\par
“对免费可得的程序打广告是做无用功。”\par
有很多免费或极其廉价的宣传形式可以用来通知计算机用户关于GNU的消息。但是使用广告可能会通知到更多的计算机用户。如果真是这样,那么通过广告收费寄送GNU拷贝的业务应该可以赚回广告费及更多。这样的话,只有从该广告获利的用户才付费。\par
另一方面,如果许多人从朋友处获得GNU,而此类业务并不成功,那么说明靠广告传播GNU并无实际必要。为什么自由市场的倡导者不能让自由市场决定这件事呢?\footnote{虽然它不是公司而只是慈善机构,自由软件基金会有10年是靠发行服务来获得其大部分资金的。你可以通过\href{http://www.gnu.org/order/order.html}{从FSF订购东西}来支持它的工作。 }\par
“我公司需要专有操作系统来在竞争中取胜。”\par
GNU将把操作系统软件从竞争的王国中移除。你不能在此取胜,你的对手也不能。你们将在其他方面竞争,但同时在操作系统领域获利。如果你的业务是销售操作系统,那么你不会喜欢GNU,但这对你来说是困难的事。如果你的业务是其他,GNU能够把你从昂贵的操作系统售价中解救出来。\par
我很想看到许多制造商和用户会捐助GNU的开发,这样会降低他们的花费。\footnote{一组公司在1991年左右集资来支持GNU C编译器的维护。}\par
“难道程序员不该因为他们的创造力得到回报吗?”\par
值得回报的东西应该是对社会的贡献。创造力可以是一种社会贡献,但只有在社会能够自由使用其结果时才是。如果程序员应该由于创新程序而得到回报,同理,他们也应该由于限制程序的使用而得到惩罚。\par
“难道程序员不能为自己的创造力要求回报吗?”\par
工作获得报酬或追求更高的薪酬并没有什么不对,只要我们不使用破坏性的手段。但是今天,软件领域的常规手段就是建立在破坏之上的。\par
因为限制减少了程序使用的方法和人数,所以通过限制程序的使用来从用户身上榨取钱财是破坏性的。它限制了人类可以从该程序中获得财富的总量。当限制是故意为之,伤害的结果就是故意破坏。\par
优秀公民不会使用这种破坏手段来致富的原因是,如果每人都这样,我们都会被相互破坏搞得更穷困。这是康德伦理\footnote{Kantian Ethics,康德伦理。是指德国哲学家康德的义务论伦理思想,其基本观点是,世界上只有一个东西是无条件的善,不但它自身是无条件善的,而且也是使一切其他东西成为善的条件,这个东西就是理性,即善良意志。};或者叫黄金定律。因为我不喜欢这样的结果,所以如果每个人都囤积信息,我就有义务说这样做是不对的。特别地,希望个人的创造力有回报并不能证明剥夺其他人的这种创造力就是对的。\par
“程序员不就饿死了吗?”\par
我可能会回答没人被迫成为程序员。我们大多数人无法靠沿街乞讨过活。但结果是,我们并没有被迫沿街乞讨并挨饿。我们会去做其他事情。\par
然而,这个回答是错的,因为它承认了提问者隐含的假设:没有软件的所有权,程序员就可能不会收到任何报酬。据此,报酬不是全部、就是没有。\par
程序员不被饿死的真正原因是他们还有能从编程谋生的方法;只是不如现在赚得多罢了。\par
限制拷贝不是软件行业唯一的基础。它是最常见的基础\footnote{我觉得我说专有软件是软件行业最常见的赚钱基础是个错误。看起来,定制软件开发过去和现在实际上都是最常见的商业模式。这个商业模式不提供收取租金的可能性,所以它必须不断地做事来维持收入。在自由软件的世界,软件定制行业还会继续存在,基本没什么变化。因此,我不再预期程序员在自由软件的世界里收入会变少。}因为它收获了最多的金钱。如果它被禁止或被客户拒绝,软件行业会迁移到那些现在不常用的基础结构之上。总是有多种方式来组织经营活动的。\par
也许在新基础之上的编程工作不再象现在一样可以赚大钱。可是那并不是反驳该变化的论据。现在销售人员按劳取酬并无不妥。如果程序员这样,那么也是正当的。(实际上,他们也许还能赚更多。)\par
“难道人们没有权利控制自己的创造力如何被使用?”\par
“控制自己想法的应用”真的构成对其他人生活的控制;而且通常是使他人的生活更困难。\par
认真研究过知识产权问题\footnote{在20世纪80年代,我还没有意识到谈论“知识产权”的“问题”多么令人困惑。该术语明显是倾向性的;较不明显的事实是,它把针对非常不同问题的多种互不相干的法律纠结在一起。现在,我敦促人们彻底拒绝“知识产权”这一术语,免得它导致其他人以为这些法律构成一个相关的问题。明确的方法应该是独立讨论专利、版权和商标。请参看关于该术语如何散布混乱和偏见的\href{http://www.gnu.org/philosophy/not-ipr.html}{进一步解释}。}的人(比如律师)会说知识产权并非天生的权利。政府确认的那些知识产权种类是有具体目的的特定法律活动的产物。\par
比如,专利体系是为了鼓励发明家公开其发明详情而建立的。其目的是帮助社会而不是帮助发明家。那时,17年的专利期相比技术进步的速度是短暂的。由于专利只是制造商之间的问题,对他们来说,专利协议的花费比生产建设要小,所以专利通常没有太大的害处。专利没有限制使用它们的大多数用户。\par
版权的概念在古代并不存在,那时作者们经常互相大量拷贝非文学类作品。这是很实用的活动,也是许多作者的作品能够哪怕只有一部分流传下来的唯一方法。版权系统为鼓励作者权益而特意创建。在其创建的发明领域—书籍,只有用印刷机才能有效拷贝—版权没什么害处,也没有限制大多数读者。\par
所有知识产权都只是社会发放的许可证,因为人们曾经认为,不管是对还是错,发放这样的许可证可以使整个社会受益。但是就任何具体情况来说,我们都要问:发放该许可证真的让我们受益了吗?获得授权的人能够从事什么活动呢?\par
今天的软件和一百年前的书籍有很大的不同。软件最容易的拷贝是人传人,软件有源代码和目标代码两种不同形式,软件是来使用而不是阅读和欣赏的,这些事实结合在一起就构成了一种情形。在此情形下,加强版权对整个社会在物质和精神上都是伤害;无论法律是否允许,我们此时都不应该再维护版权。\par
“竞争使东西变得更好。”\par
赛跑是竞争的典范:通过回报优胜者,我们鼓励人们跑得更快。当资本主义真的这样运作时,它做得很好;但是其辩护者做的这个假设并不总是对的。如果竞争者忘记了回报的原因而只想着胜利,不计方法,那么他们就可能使用其他的策略—比如攻击别的竞争者。如果竞争者在互相打架,大家就都跑不快。\par
专有软件和保密软件在道德上等同于互相打架的竞争者。令人沮丧的是,我们唯一的裁判看来并不反对打架;他只是规范打架者(“每跑10米,你们可以打一下”)。他真的应该把他们分开,并严惩试图打架的竞争者。\par
“没有金钱刺激,人们不就不再编程了吗?”\par
实际上,许多人在绝对没有金钱刺激的情况下也会编程。编程对一些人有不可抗拒的魔力,这些人往往是最擅长编程的那些人。从来也不缺少坚持音乐的职业音乐家,即使他们毫无希望靠音乐谋生。\par
但是这个问题,虽然经常被问到,并不是指这种情况。程序员会得到报酬,只是变少。所以问题应该是,金钱减少时,还有人编程吗?我的经验是:有。\par
10多年来,许多世界上最好的程序员在人工智能实验室工作,这里的收入要比他们到其他地方工作少得太多。他们获得了许多非金钱的回报:比如,名望和感谢。而创造力本身也是快乐,也是回报。\par
然后,当有机会做同样有趣的工作并赚大钱时,大多数人离开了。\par
这说明人们会为致富之外的理由编程;如果有同时也能赚到大钱的机会,他们也会选择它。薪水低的企业在和薪水高的企业竞争时表现不佳,但是如果薪水高的企业被禁止,低薪水的企业不应该再表现差劲吧。\par
“我们迫切需要程序员。如果他们要求我们不要帮助友邻,我们不得不那样做。”\par
你永远也不会绝望到去遵守这样的命令。请记住:宁为玉碎,不为瓦全!\footnote{Millions for defense, but not a cent for tribute!原意是宁可战斗,也不乞和!}\par
“程序员也需要谋生啊。”\par
短期来看,是这样的。不过,程序员有很多不用出卖程序的使用权利就可以谋生的方法。出卖权利现在成了惯例,是因为它带给程序员和生意人最多的钱财,而不是因为它是谋生的唯一手段。如果想要,我们能够轻易找到其他的方法。这里举几个例子。\par
制造商新引进新计算机需要雇人来把操作系统移植到新硬件上。\par
教育培训、手把手服务和维护服务也可能雇佣程序员。\par
有新想法的人可以发布免费软件\footnote{后来,我们了解到要区别“自由软件”和“免费软件”。“免费软件”是指你可以自由再发布的软件,但是你并没有自由来学习和修改其源代码,所以大部分免费软件不是自由软件。请参看\href{http://www.gnu.org/philosophy/words-to-avoid.html\#GiveAwaySoftware}{“不清楚的词汇和短语”}了解更多解释。},并向对此满意的用户寻求捐助,或者是销售手把手服务。我就碰到一些成功这样做的人。\par
需求相关的用户可以组建用户组,并支付会费。用户组就可以和程序公司签约让公司定制组内成员需要的程序。\par
所有开发费用都可以由软件税来支付:\par
假设每个购买计算机的用户都要按价格支付一定比例的软件税。政府可以让诸如NSF \footnote{NSF, National Science Foundation:美国国家科学基金会。}之类的代理使用该税收支持软件开发。\par
但是如果购买者自己向软件开发做了捐助,那么他可以减税。他可以自己选择捐助项目—通常,他会选择他希望能够用到的项目。减税额度最高是免税。\par
税率可以由交税的人投票决定,票的权重可以按大家的应税额来算。\par
结果:\par
计算机使用社群支持软件开发。\par
该社群决定应该支持到什么程度。\par
用户可以根据自己的需要来选择要支持的项目。\par
长远来看,让软件自由是通往富足世界的一小步;在富足世界里,人们不必辛苦工作来谋生。人们在每周10小时的法律活动、家庭咨询、机器人维修和流星观察等规定任务之外,能够自由投入到象编程这样的有趣活动中。那时,就没有必要再以编程为谋生手段了。\par
我们已经把整个社会要维持生产力的工作大大减少了,但是只有很少一部分转化为劳动者的闲暇,因为生产活动需要夹杂很多的非生产活动。其主要原因是官僚主义和对竞争的抗拒。自由软件会大大减少在软件生产领域的生产力流失。我们必须做这件事,为了使技术进步带来的生产力提高能够转化为人们工作的减少。
\chapter{关于本书}
在这里你能读到一些关于本书的信息。
\section{版本历史}
开始编写:January 28, 2019\par
编译日期:\today \par
使用\XeLaTeX 编译,\BIBTEX 管理文献。
试行版
\section{我该如何对这个工程做出贡献?}
请将更改的内容或需要反馈的错误写成一个txt文件,它应该像这样子:
\begin{verbatim}
COM:https://github.com/sgcomputersLab/COM_Zh_CN
[版本号]
[错误性质]
[详细内容]
##解释:
##[版本号]应为所使用的提交的哈希值(如156ba48adc942a41a219a65752187bcb7981b99b)。
##[错误性质] 应为以下内容中的任意一个:
##[REF]引用错误;侵权
##此时[详细内容]就应该包括具体的内容,在整个PDF文件中的位置,正确的来源
##以及是否可以在修正后继续使用。
##[CONT]内容错误
##此时[详细内容]就应该包括具体的内容,在整个PDF文件中的位置,正确的内容。
##[COMP]编译错误
##此时[详细内容]可省略。
\end{verbatim} \par
例如:
\begin{verbatim}
COM:https://github.com/sgcomputersLab/COM_Zh_CN
[CONT]
commit 156ba48adc942a41a219a65752187bcb7981b99b
内容略。
\end{verbatim} \par
目前YuZJ管理能力有限,因此请不要fork此项目或直接推送更改。
\section{-感谢}
\section{商标确认}
所有商标的所有权归各自的商标所有者。
\section{编译方法}
对于Windows操作系统,请按以下步骤操作:\par
1.确保您的计算机操作系统为Windows7 sp1及以上。安装TeXLive或2019,并确保“C:/texlive/2019/bin/Win32/ ”目录(这是默认安装路径)已被添加到PATH环境变量中,安装并更新所有宏包。\par
2.运行Compile.bat(在一般的Windows系统下双击即可),它应位于源码包的根目录内。\par
3.如果您希望手动编译,请在根目录内打开命令提示符,按顺序输入如下命令:
\begin{verbatim*}
xelatex Main
bibtex Main
xelatex Main
xelatex Main
del *.log
del *.aux
del *.bbl
del *.blg
del *.toc
del *.out
del *.fls
del *.fdb_latexmk
del *.syntex.gz
\end{verbatim*}
对于GNU/Linux操作系统,请按以下步骤操作:\par
1.安装TeXLive 2019,并确保“/usr/local/texlive/2019/bin/x86\_64-linux/ ”目录(这是默认安装路径)已被添加到PATH环境变量中,安装且更新所有宏包。\par
2.运行Compile.sh(在目录内打开(伪)终端,输入“./Compile.sh”。可能需要先运行“chmod +x Complie.sh”),它应位于源码包的TeX目录内。\par
3.如果您希望手动编译,请在TeX目录内打开(伪)终端,按顺序输入如下命令:
\begin{verbatim*}
xelatex Main.TeX
bibtex Main
xelatex Main.TeX
xelatex Main.TeX
rm -f *.log
rm -f *.aux
rm -f *.bbl
rm -f *.blg
rm -f *.toc
rm -f *.out
rm -f *.fls
rm -f *.fdb_latexmk
\end{verbatim*}
无论你使用何种编译方式,你最终将会得到Main.PDF。这就是最终的文件。
\section{关于作者}
YuZJ,浙江省慈溪市实验高中2019届学生,电教委员。现就读于浙江大学-爱丁堡大学联合学院生物信息学系。
\chapter{GNU GENERAL PUBLIC LICENSE}
\begin{center} Version 3, 29 June 2007 \end{center}
\footnote{请注意,本书中引用《GNU GPL》仅仅作为附件,本书的真正许可证是《署名-非商业性使用-相同方式共享 3.0中国大陆》。}\cite{gpl}Copyright \copyright 2007 Free Software Foundation, Inc. <\url{https://fsf.org/}>\par
Everyone is permitted to copy and distribute verbatim copies of this license document, but changing it is not allowed.\par
\section{Preamble}
The GNU General Public License is a free, copyleft license for software and other kinds of works.\par
The licenses for most software and other practical works are designed to take away your freedom to share and change the works. By contrast, the GNU General Public License is intended to guarantee your freedom to share and change all versions of a program--to make sure it remains free software for all its users. We, the Free Software Foundation, use the GNU General Public License for most of our software; it applies also to any other work released this way by its authors. You can apply it to your programs, too.\par
When we speak of free software, we are referring to freedom, not price. Our General Public Licenses are designed to make sure that you have the freedom to distribute copies of free software (and charge for them if you wish), that you receive source code or can get it if you want it, that you can change the software or use pieces of it in new free programs, and that you know you can do these things.\par
To protect your rights, we need to prevent others from denying you these rights or asking you to surrender the rights. Therefore, you have certain responsibilities if you distribute copies of the software, or if you modify it: responsibilities to respect the freedom of others.\par
For example, if you distribute copies of such a program, whether gratis or for a fee, you must pass on to the recipients the same freedoms that you received. You must make sure that they, too, receive or can get the source code. And you must show them these terms so they know their rights.\par
Developers that use the GNU GPL protect your rights with two steps: (1) assert copyright on the software, and (2) offer you this License giving you legal permission to copy, distribute and/or modify it.\par
For the developers' and authors' protection, the GPL clearly explains that there is no warranty for this free software. For both users' and authors' sake, the GPL requires that modified versions be marked as changed, so that their problems will not be attributed erroneously to authors of previous versions.\par
Some devices are designed to deny users access to install or run modified versions of the software inside them, although the manufacturer can do so. This is fundamentally incompatible with the aim of protecting users' freedom to change the software. The systematic pattern of such abuse occurs in the area of products for individuals to use, which is precisely where it is most unacceptable. Therefore, we have designed this version of the GPL to prohibit the practice for those products. If such problems arise substantially in other domains, we stand ready to extend this provision to those domains in future versions of the GPL, as needed to protect the freedom of users.\par
Finally, every program is threatened constantly by software patents. States should not allow patents to restrict development and use of software on general-purpose computers, but in those that do, we wish to avoid the special danger that patents applied to a free program could make it effectively proprietary. To prevent this, the GPL assures that patents cannot be used to render the program non-free.\par
The precise terms and conditions for copying, distribution and modification follow.
\section{TERMS AND CONDITIONS}
\subsection{0. Definitions.}
“This License” refers to version 3 of the GNU General Public License.\par
“Copyright” also means copyright-like laws that apply to other kinds of works, such as semiconductor masks.\par
“The Program” refers to any copyrightable work licensed under this License. Each licensee is addressed as “you”. “Licensees” and “recipients” may be individuals or organizations.\par
To “modify” a work means to copy from or adapt all or part of the work in a fashion requiring copyright permission, other than the making of an exact copy. The resulting work is called a “modified version” of the earlier work or a work “based on” the earlier work.\par
A “covered work” means either the unmodified Program or a work based on the Program.\par
To “propagate” a work means to do anything with it that, without permission, would make you directly or secondarily liable for infringement under applicable copyright law, except executing it on a computer or modifying a private copy. Propagation includes copying, distribution (with or without modification), making available to the public, and in some countries other activities as well.\par
To “convey” a work means any kind of propagation that enables other parties to make or receive copies. Mere interaction with a user through a computer network, with no transfer of a copy, is not conveying.\par
An interactive user interface displays “Appropriate Legal Notices” to the extent that it includes a convenient and prominently visible feature that (1) displays an appropriate copyright notice, and (2) tells the user that there is no warranty for the work (except to the extent that warranties are provided), that licensees may convey the work under this License, and how to view a copy of this License. If the interface presents a list of user commands or options, such as a menu, a prominent item in the list meets this criterion.
\subsection{1. Source Code.}
The “source code” for a work means the preferred form of the work for making modifications to it. “Object code” means any non-source form of a work.\par
A “Standard Interface” means an interface that either is an official standard defined by a recognized standards body, or, in the case of interfaces specified for a particular programming language, one that is widely used among developers working in that language.\par
The “System Libraries” of an executable work include anything, other than the work as a whole, that (a) is included in the normal form of packaging a Major Component, but which is not part of that Major Component, and (b) serves only to enable use of the work with that Major Component, or to implement a Standard Interface for which an implementation is available to the public in source code form. A “Major Component”, in this context, means a major essential component (kernel, window system, and so on) of the specific operating system (if any) on which the executable work runs, or a compiler used to produce the work, or an object code interpreter used to run it.\par
The “Corresponding Source” for a work in object code form means all the source code needed to generate, install, and (for an executable work) run the object code and to modify the work, including scripts to control those activities. However, it does not include the work's System Libraries, or general-purpose tools or generally available free programs which are used unmodified in performing those activities but which are not part of the work. For example, Corresponding Source includes interface definition files associated with source files for the work, and the source code for shared libraries and dynamically linked subprograms that the work is specifically designed to require, such as by intimate data communication or control flow between those subprograms and other parts of the work.\par
The Corresponding Source need not include anything that users can regenerate automatically from other parts of the Corresponding Source.\par
The Corresponding Source for a work in source code form is that same work.
\subsection{2. Basic Permissions.}
All rights granted under this License are granted for the term of copyright on the Program, and are irrevocable provided the stated conditions are met. This License explicitly affirms your unlimited permission to run the unmodified Program. The output from running a covered work is covered by this License only if the output, given its content, constitutes a covered work. This License acknowledges your rights of fair use or other equivalent, as provided by copyright law.\par
You may make, run and propagate covered works that you do not convey, without conditions so long as your license otherwise remains in force. You may convey covered works to others for the sole purpose of having them make modifications exclusively for you, or provide you with facilities for running those works, provided that you comply with the terms of this License in conveying all material for which you do not control copyright. Those thus making or running the covered works for you must do so exclusively on your behalf, under your direction and control, on terms that prohibit them from making any copies of your copyrighted material outside their relationship with you.\par
Conveying under any other circumstances is permitted solely under the conditions stated below. Sublicensing is not allowed; section 10 makes it unnecessary.
\subsection{3. Protecting Users' Legal Rights From Anti-Circumvention Law.}
No covered work shall be deemed part of an effective technological measure under any applicable law fulfilling obligations under article 11 of the WIPO copyright treaty adopted on 20 December 1996, or similar laws prohibiting or restricting circumvention of such measures.\par
When you convey a covered work, you waive any legal power to forbid circumvention of technological measures to the extent such circumvention is effected by exercising rights under this License with respect to the covered work, and you disclaim any intention to limit operation or modification of the work as a means of enforcing, against the work's users, your or third parties' legal rights to forbid circumvention of technological measures.
\subsection{4. Conveying Verbatim Copies.}
You may convey verbatim copies of the Program's source code as you receive it, in any medium, provided that you conspicuously and appropriately publish on each copy an appropriate copyright notice; keep intact all notices stating that this License and any non-permissive terms added in accord with section 7 apply to the code; keep intact all notices of the absence of any warranty; and give all recipients a copy of this License along with the Program.\par
You may charge any price or no price for each copy that you convey, and you may offer support or warranty protection for a fee.
\subsection{5. Conveying Modified Source Versions.}
You may convey a work based on the Program, or the modifications to produce it from the Program, in the form of source code under the terms of section 4, provided that you also meet all of these conditions:\par
a) The work must carry prominent notices stating that you modified it, and giving a relevant date.\par
b) The work must carry prominent notices stating that it is released under this License and any conditions added under section 7. This requirement modifies the requirement in section 4 to “keep intact all notices”.\par
c) You must license the entire work, as a whole, under this License to anyone who comes into possession of a copy. This License will therefore apply, along with any applicable section 7 additional terms, to the whole of the work, and all its parts, regardless of how they are packaged. This License gives no permission to license the work in any other way, but it does not invalidate such permission if you have separately received it.\par
d) If the work has interactive user interfaces, each must display Appropriate Legal Notices; however, if the Program has interactive interfaces that do not display Appropriate Legal Notices, your work need not make them do so.\par
A compilation of a covered work with other separate and independent works, which are not by their nature extensions of the covered work, and which are not combined with it such as to form a larger program, in or on a volume of a storage or distribution medium, is called an “aggregate” if the compilation and its resulting copyright are not used to limit the access or legal rights of the compilation's users beyond what the individual works permit. Inclusion of a covered work in an aggregate does not cause this License to apply to the other parts of the aggregate.
\subsection{6. Conveying Non-Source Forms.}
You may convey a covered work in object code form under the terms of sections 4 and 5, provided that you also convey the machine-readable Corresponding Source under the terms of this License, in one of these ways:\par
a) Convey the object code in, or embodied in, a physical product (including a physical distribution medium), accompanied by the Corresponding Source fixed on a durable physical medium customarily used for software interchange.\par
b) Convey the object code in, or embodied in, a physical product (including a physical distribution medium), accompanied by a written offer, valid for at least three years and valid for as long as you offer spare parts or customer support for that product model, to give anyone who possesses the object code either (1) a copy of the Corresponding Source for all the software in the product that is covered by this License, on a durable physical medium customarily used for software interchange, for a price no more than your reasonable cost of physically performing this conveying of source, or (2) access to copy the Corresponding Source from a network server at no charge.\par
c) Convey individual copies of the object code with a copy of the written offer to provide the Corresponding Source. This alternative is allowed only occasionally and noncommercially, and only if you received the object code with such an offer, in accord with subsection 6b.\par
d) Convey the object code by offering access from a designated place (gratis or for a charge), and offer equivalent access to the Corresponding Source in the same way through the same place at no further charge. You need not require recipients to copy the Corresponding Source along with the object code. If the place to copy the object code is a network server, the Corresponding Source may be on a different server (operated by you or a third party) that supports equivalent copying facilities, provided you maintain clear directions next to the object code saying where to find the Corresponding Source. Regardless of what server hosts the Corresponding Source, you remain obligated to ensure that it is available for as long as needed to satisfy these requirements.\par
e) Convey the object code using peer-to-peer transmission, provided you inform other peers where the object code and Corresponding Source of the work are being offered to the general public at no charge under subsection 6d.\par
A separable portion of the object code, whose source code is excluded from the Corresponding Source as a System Library, need not be included in conveying the object code work.
A “User Product” is either (1) a “consumer product”, which means any tangible personal property which is normally used for personal, family, or household purposes, or (2) anything designed or sold for incorporation into a dwelling. In determining whether a product is a consumer product, doubtful cases shall be resolved in favor of coverage. For a particular product received by a particular user, “normally used” refers to a typical or common use of that class of product, regardless of the status of the particular user or of the way in which the particular user actually uses, or expects or is expected to use, the product. A product is a consumer product regardless of whether the product has substantial commercial, industrial or non-consumer uses, unless such uses represent the only significant mode of use of the product.
“Installation Information” for a User Product means any methods, procedures, authorization keys, or other information required to install and execute modified versions of a covered work in that User Product from a modified version of its Corresponding Source. The information must suffice to ensure that the continued functioning of the modified object code is in no case prevented or interfered with solely because modification has been made.\par
If you convey an object code work under this section in, or with, or specifically for use in, a User Product, and the conveying occurs as part of a transaction in which the right of possession and use of the User Product is transferred to the recipient in perpetuity or for a fixed term (regardless of how the transaction is characterized), the Corresponding Source conveyed under this section must be accompanied by the Installation Information. But this requirement does not apply if neither you nor any third party retains the ability to install modified object code on the User Product (for example, the work has been installed in ROM).\par
The requirement to provide Installation Information does not include a requirement to continue to provide support service, warranty, or updates for a work that has been modified or installed by the recipient, or for the User Product in which it has been modified or installed. Access to a network may be denied when the modification itself materially and adversely affects the operation of the network or violates the rules and protocols for communication across the network.\par
Corresponding Source conveyed, and Installation Information provided, in accord with this section must be in a format that is publicly documented (and with an implementation available to the public in source code form), and must require no special password or key for unpacking, reading or copying.
\subsection{7. Additional Terms.}
“Additional permissions” are terms that supplement the terms of this License by making exceptions from one or more of its conditions. Additional permissions that are applicable to the entire Program shall be treated as though they were included in this License, to the extent that they are valid under applicable law. If additional permissions apply only to part of the Program, that part may be used separately under those permissions, but the entire Program remains governed by this License without regard to the additional permissions.\par
When you convey a copy of a covered work, you may at your option remove any additional permissions from that copy, or from any part of it. (Additional permissions may be written to require their own removal in certain cases when you modify the work.) You may place additional permissions on material, added by you to a covered work, for which you have or can give appropriate copyright permission.\par
Notwithstanding any other provision of this License, for material you add to a covered work, you may (if authorized by the copyright holders of that material) supplement the terms of this License with terms:\par
a) Disclaiming warranty or limiting liability differently from the terms of sections 15 and 16 of this License; or\par
b) Requiring preservation of specified reasonable legal notices or author attributions in that material or in the Appropriate Legal Notices displayed by works containing it; or\par
c) Prohibiting misrepresentation of the origin of that material, or requiring that modified versions of such material be marked in reasonable ways as different from the original version; or\par
d) Limiting the use for publicity purposes of names of licensors or authors of the material; or\par
e) Declining to grant rights under trademark law for use of some trade names, trademarks, or service marks; or\par
f) Requiring indemnification of licensors and authors of that material by anyone who conveys the material (or modified versions of it) with contractual assumptions of liability to the recipient, for any liability that these contractual assumptions directly impose on those licensors and authors.\par
All other non-permissive additional terms are considered “further restrictions” within the meaning of section 10. If the Program as you received it, or any part of it, contains a notice stating that it is governed by this License along with a term that is a further restriction, you may remove that term. If a license document contains a further restriction but permits relicensing or conveying under this License, you may add to a covered work material governed by the terms of that license document, provided that the further restriction does not survive such relicensing or conveying.\par
If you add terms to a covered work in accord with this section, you must place, in the relevant source files, a statement of the additional terms that apply to those files, or a notice indicating where to find the applicable terms.\par
Additional terms, permissive or non-permissive, may be stated in the form of a separately written license, or stated as exceptions; the above requirements apply either way.
\subsection{8. Termination.}
You may not propagate or modify a covered work except as expressly provided under this License. Any attempt otherwise to propagate or modify it is void, and will automatically terminate your rights under this License (including any patent licenses granted under the third paragraph of section 11).\par
However, if you cease all violation of this License, then your license from a particular copyright holder is reinstated (a) provisionally, unless and until the copyright holder explicitly and finally terminates your license, and (b) permanently, if the copyright holder fails to notify you of the violation by some reasonable means prior to 60 days after the cessation.\par
Moreover, your license from a particular copyright holder is reinstated permanently if the copyright holder notifies you of the violation by some reasonable means, this is the first time you have received notice of violation of this License (for any work) from that copyright holder, and you cure the violation prior to 30 days after your receipt of the notice.\par
Termination of your rights under this section does not terminate the licenses of parties who have received copies or rights from you under this License. If your rights have been terminated and not permanently reinstated, you do not qualify to receive new licenses for the same material under section 10.
\subsection{9. Acceptance Not Required for Having Copies.}
You are not required to accept this License in order to receive or run a copy of the Program. Ancillary propagation of a covered work occurring solely as a consequence of using peer-to-peer transmission to receive a copy likewise does not require acceptance. However, nothing other than this License grants you permission to propagate or modify any covered work. These actions infringe copyright if you do not accept this License. Therefore, by modifying or propagating a covered work, you indicate your acceptance of this License to do so.
\subsection{10. Automatic Licensing of Downstream Recipients.}
Each time you convey a covered work, the recipient automatically receives a license from the original licensors, to run, modify and propagate that work, subject to this License. You are not responsible for enforcing compliance by third parties with this License.\par
An “entity transaction” is a transaction transferring control of an organization, or substantially all assets of one, or subdividing an organization, or merging organizations. If propagation of a covered work results from an entity transaction, each party to that transaction who receives a copy of the work also receives whatever licenses to the work the party's predecessor in interest had or could give under the previous paragraph, plus a right to possession of the Corresponding Source of the work from the predecessor in interest, if the predecessor has it or can get it with reasonable efforts.\par
You may not impose any further restrictions on the exercise of the rights granted or affirmed under this License. For example, you may not impose a license fee, royalty, or other charge for exercise of rights granted under this License, and you may not initiate litigation (including a cross-claim or counterclaim in a lawsuit) alleging that any patent claim is infringed by making, using, selling, offering for sale, or importing the Program or any portion of it.
\subsection{11. Patents.}
A “contributor” is a copyright holder who authorizes use under this License of the Program or a work on which the Program is based. The work thus licensed is called the contributor's “contributor version”.\par
A contributor's “essential patent claims” are all patent claims owned or controlled by the contributor, whether already acquired or hereafter acquired, that would be infringed by some manner, permitted by this License, of making, using, or selling its contributor version, but do not include claims that would be infringed only as a consequence of further modification of the contributor version. For purposes of this definition, “control” includes the right to grant patent sublicenses in a manner consistent with the requirements of this License.\par
Each contributor grants you a non-exclusive, worldwide, royalty-free patent license under the contributor's essential patent claims, to make, use, sell, offer for sale, import and otherwise run, modify and propagate the contents of its contributor version.\par
In the following three paragraphs, a “patent license” is any express agreement or commitment, however denominated, not to enforce a patent (such as an express permission to practice a patent or covenant not to sue for patent infringement). To “grant” such a patent license to a party means to make such an agreement or commitment not to enforce a patent against the party.\par
If you convey a covered work, knowingly relying on a patent license, and the Corresponding Source of the work is not available for anyone to copy, free of charge and under the terms of this License, through a publicly available network server or other readily accessible means, then you must either (1) cause the Corresponding Source to be so available, or (2) arrange to deprive yourself of the benefit of the patent license for this particular work, or (3) arrange, in a manner consistent with the requirements of this License, to extend the patent license to downstream recipients. “Knowingly relying” means you have actual knowledge that, but for the patent license, your conveying the covered work in a country, or your recipient's use of the covered work in a country, would infringe one or more identifiable patents in that country that you have reason to believe are valid.\par
If, pursuant to or in connection with a single transaction or arrangement, you convey, or propagate by procuring conveyance of, a covered work, and grant a patent license to some of the parties receiving the covered work authorizing them to use, propagate, modify or convey a specific copy of the covered work, then the patent license you grant is automatically extended to all recipients of the covered work and works based on it.\par
A patent license is “discriminatory” if it does not include within the scope of its coverage, prohibits the exercise of, or is conditioned on the non-exercise of one or more of the rights that are specifically granted under this License. You may not convey a covered work if you are a party to an arrangement with a third party that is in the business of distributing software, under which you make payment to the third party based on the extent of your activity of conveying the work, and under which the third party grants, to any of the parties who would receive the covered work from you, a discriminatory patent license (a) in connection with copies of the covered work conveyed by you (or copies made from those copies), or (b) primarily for and in connection with specific products or compilations that contain the covered work, unless you entered into that arrangement, or that patent license was granted, prior to 28 March 2007.\par
Nothing in this License shall be construed as excluding or limiting any implied license or other defenses to infringement that may otherwise be available to you under applicable patent law.
\subsection{12. No Surrender of Others' Freedom.}
If conditions are imposed on you (whether by court order, agreement or otherwise) that contradict the conditions of this License, they do not excuse you from the conditions of this License. If you cannot convey a covered work so as to satisfy simultaneously your obligations under this License and any other pertinent obligations, then as a consequence you may not convey it at all. For example, if you agree to terms that obligate you to collect a royalty for further conveying from those to whom you convey the Program, the only way you could satisfy both those terms and this License would be to refrain entirely from conveying the Program.
\subsection{13. Use with the GNU Affero General Public License.}
Notwithstanding any other provision of this License, you have permission to link or combine any covered work with a work licensed under version 3 of the GNU Affero General Public License into a single combined work, and to convey the resulting work. The terms of this License will continue to apply to the part which is the covered work, but the special requirements of the GNU Affero General Public License, section 13, concerning interaction through a network will apply to the combination as such.
\subsection{14. Revised Versions of this License.}
The Free Software Foundation may publish revised and/or new versions of the GNU General Public License from time to time. Such new versions will be similar in spirit to the present version, but may differ in detail to address new problems or concerns.\par
Each version is given a distinguishing version number. If the Program specifies that a certain numbered version of the GNU General Public License “or any later version” applies to it, you have the option of following the terms and conditions either of that numbered version or of any later version published by the Free Software Foundation. If the Program does not specify a version number of the GNU General Public License, you may choose any version ever published by the Free Software Foundation.\par
If the Program specifies that a proxy can decide which future versions of the GNU General Public License can be used, that proxy's public statement of acceptance of a version permanently authorizes you to choose that version for the Program.\par
Later license versions may give you additional or different permissions. However, no additional obligations are imposed on any author or copyright holder as a result of your choosing to follow a later version.
\subsection{15. Disclaimer of Warranty.}
THERE IS NO WARRANTY FOR THE PROGRAM, TO THE EXTENT PERMITTED BY APPLICABLE LAW. EXCEPT WHEN OTHERWISE STATED IN WRITING THE COPYRIGHT HOLDERS AND/OR OTHER PARTIES PROVIDE THE PROGRAM “AS IS” WITHOUT WARRANTY OF ANY KIND, EITHER EXPRESSED OR IMPLIED, INCLUDING, BUT NOT LIMITED TO, THE IMPLIED WARRANTIES OF MERCHANTABILITY AND FITNESS FOR A PARTICULAR PURPOSE. THE ENTIRE RISK AS TO THE QUALITY AND PERFORMANCE OF THE PROGRAM IS WITH YOU. SHOULD THE PROGRAM PROVE DEFECTIVE, YOU ASSUME THE COST OF ALL NECESSARY SERVICING, REPAIR OR CORRECTION.
\subsection{16. Limitation of Liability.}
IN NO EVENT UNLESS REQUIRED BY APPLICABLE LAW OR AGREED TO IN WRITING WILL ANY COPYRIGHT HOLDER, OR ANY OTHER PARTY WHO MODIFIES AND/OR CONVEYS THE PROGRAM AS PERMITTED ABOVE, BE LIABLE TO YOU FOR DAMAGES, INCLUDING ANY GENERAL, SPECIAL, INCIDENTAL OR CONSEQUENTIAL DAMAGES ARISING OUT OF THE USE OR INABILITY TO USE THE PROGRAM (INCLUDING BUT NOT LIMITED TO LOSS OF DATA OR DATA BEING RENDERED INACCURATE OR LOSSES SUSTAINED BY YOU OR THIRD PARTIES OR A FAILURE OF THE PROGRAM TO OPERATE WITH ANY OTHER PROGRAMS), EVEN IF SUCH HOLDER OR OTHER PARTY HAS BEEN ADVISED OF THE POSSIBILITY OF SUCH DAMAGES.
\subsection{17. Interpretation of Sections 15 and 16.}
If the disclaimer of warranty and limitation of liability provided above cannot be given local legal effect according to their terms, reviewing courts shall apply local law that most closely approximates an absolute waiver of all civil liability in connection with the Program, unless a warranty or assumption of liability accompanies a copy of the Program in return for a fee.
\begin{center}END OF TERMS AND CONDITIONS\end{center}
\section{How to Apply These Terms to Your New Programs}
If you develop a new program, and you want it to be of the greatest possible use to the public, the best way to achieve this is to make it free software which everyone can redistribute and change under these terms.\par
To do so, attach the following notices to the program. It is safest to attach them to the start of each source file to most effectively state the exclusion of warranty; and each file should have at least the “copyright” line and a pointer to where the full notice is found.
\begin{verbatim}
<one line to give the program's name and a brief idea of what it does.>
Copyright (C) <year>  <name of author>

This program is free software: you can redistribute it and/or modify
it under the terms of the GNU General Public License as published by
the Free Software Foundation, either version 3 of the License, or
(at your option) any later version.

This program is distributed in the hope that it will be useful,
but WITHOUT ANY WARRANTY; without even the implied warranty of
MERCHANTABILITY or FITNESS FOR A PARTICULAR PURPOSE.  See the
GNU General Public License for more details.

You should have received a copy of the GNU General Public License
along with this program.  If not, see <https://www.gnu.org/licenses/>.
\end{verbatim}\par
Also add information on how to contact you by electronic and paper mail.\par
If the program does terminal interaction, make it output a short notice like this when it starts in an interactive mode:
\begin{verbatim}
<program>  Copyright (C) <year>  <name of author>
This program comes with ABSOLUTELY NO WARRANTY; for details type `show w'.
This is free software, and you are welcome to redistribute it
under certain conditions; type `show c' for details.
\end{verbatim}\par
The hypothetical commands `show w' and `show c' should show the appropriate parts of the General Public License. Of course, your program's commands might be different; for a GUI interface, you would use an “about box”.\par
You should also get your employer (if you work as a programmer) or school, if any, to sign a “copyright disclaimer” for the program, if necessary. For more information on this, and how to apply and follow the GNU GPL, see <\url{https://www.gnu.org/licenses/}>.\par
The GNU General Public License does not permit incorporating your program into proprietary programs. If your program is a subroutine library, you may consider it more useful to permit linking proprietary applications with the library. If this is what you want to do, use the GNU Lesser General Public License instead of this License. But first, please read <\url{https://www.gnu.org/licenses/why-not-lgpl.html}>.
\chapter{GNU通用公共授權合約}
\footnote{我们暂时没有找到简体中文译本,因此使用繁体中文作为替代品。当简体中文被找到时将会使用简体中文。}\cite{gplzht}第三版,2007年6月29日\par
版權所有 © 2007 自由軟體基金會 <\url{http://fsf.org/}>\par
任何人皆可複製和發佈本協議的完整副本,但不得修改\par
\section{【譯者聲明】}
This is an unofficial translation of the GNU General Public License into Chinese. It was not published by the Free Software Foundation, and does not legally state the distribution terms for software that uses the GNU GPL--only the original English text of the GNU GPL does that. However, we hope that this translation will help Chinese speakers understand the GNU GPL better. You may publish this translation, modified or unmodified, only under the terms at \url{http://www.gnu.org/licenses/translations.html}\par
這是GNU通用公共授權合約的一份非官方中文翻譯,並非自由軟體基金會所發表,不適用於使用GNU通用公共授權合約發佈的軟體的法律聲明——只有GNU通用公共授權合約英文原版才具有法律效力。不過我們希望本翻譯能夠幫助中文讀者更好地理解GNU通用公共授權合約。 您可以僅根據http://www.gnu.org/licenses/translations.html中的條款發布此修改或未修改的翻譯。
\section{【引言】}
GNU通用公共授權合約是一份面向軟體及其他類型作品的,自由的版權共產協定。\par
就多數軟體而言,授權合約被設計用於剝奪你分享和修改軟體的自由。相反,GNU通用公共授權合約力圖保障你分享和修改某程式全部版本的權利——確保自由軟體對其使用者來說是自由的。我們自由軟體基金會將GNU通用公共授權合約用於我們的大多數軟體,並為一些其他作品的作者效仿。你也可以將本協定用於你的程式。\par
所謂自由軟體,強調自由,而非免費。本GNU通用公共授權合約設計用於確保你享有分發自由軟體的自由(你可以為此服務收費),確保你可以在需要的時候獲得這些軟體的源碼,確保你可以修改這些軟體或者在新的自由軟體中複用其中某些片段,並且確保你在這方面享有知情權。\par
為保障你的權益,我們需要作一些限定:禁止任何人否認你的上述權利,或者要求你放棄它們。因此,當你分發或修改這些軟體時,你有一定的責任——尊重他人的自由。如果你分發這種程式的副本,無論收費還是免費,你必須給予與你同等的權利。你還要確保他們也能收到源碼並瞭解他們的權利。\par
採用GNU通用公共授權合約的開發者通過兩步保障你的權益:其一,申明軟體的版權;其二,通過本協議使你可以合法地複製、分發和修改該軟體。\par
為了保護每一位作者和開發者,GNU通用公共授權合約指明一點:自由軟體並沒有品質擔保。為用戶和作者雙方著想,GNU通用公共授權合約要求修改版必須有標記,以免其問題被錯誤地歸到先前版本的作者身上。\par
某些設備設計成拒絕使用者安裝運行修改過的軟體,但廠商不受限。這和我們保護使用者享有修改軟體的自由的宗旨存在根本性矛盾。該濫用協定的模式出現於個人用品領域,這恰是最不可接受的。因此,我們設計了這版GNU通用公共授權合約來禁止這類產品。如果此類問題在其他領域湧現,我們時刻準備著在將來的版本中把規定擴展到相應領域,以保護用戶的自由。\par
最後,每個程式都持續受到軟體專利的威脅。政府不應該允許專利限制通用電腦軟體的開發和應用,在做不到這點時,我們希望避免專利應用有效地使自由軟體私有化的危險。就此,GNU通用公共授權合約保證專利不能使程式非自由化。\par
下文是關於複製、分發和修改的嚴謹描述和實施條件。
\section{【關於複製、分發和修改的術語和條件】}
\subsection{〇、定義}
“本協議”指GNU通用公共授權合約第三版。\par
“版權”也指適用于諸如半導體掩模的其他類型作品的類似法律。\par
“本程式”指任何在本協議保護下的有版權的作品。每個許可獲得者稱作“你”。“許可獲得者”和“接收者”可以是個人或組織。\par
“修改”一個作品指需要版權許可的的複製及對作品全面的全面或部分改編行為,有別於製作副本。所產生的作品稱作前作的“修改版”,或“基於”前作的作品。\par
“受保護作品”指程式或其派生作品。\par
“傳播”作品指那些未經許可就會在適用版權法律下構成直接或間接侵權的行為,不包括在電腦上運行和私下的修改。傳播包括複製、分發(無論修改與否)、向公眾公開,以及在某些國家的其他行為。\par
“轉發”作品指讓他方能夠製作或者接收副本的行為。僅僅通過電腦網路和使用者交互,沒有傳輸副本,則不算轉發。\par
一個顯示“適當的法律聲明”的互動式使用者介面應包括一個便捷而醒目的視覺化特性:(1)顯示適當的版權聲明;(2)告知用戶沒有品質擔保(提供了品質擔保的情況除外),許可獲得者可以在本協定約束下轉發該作品,及查看本協定副本的途徑。如果該介面提供一個命令清單,如功能表,其表項應符合上述規範。
\subsection{一、源碼}
作品的源碼指其可修改的首選形式,目的碼指所有其他形式。\par
“標準介面”指標準化組織定義的官方標準中的介面,或針為某種程式設計語言設定的介面中為開發者廣泛使用的介面。\par
可執行作品中的“系統庫”不是指整個程式,而是涵蓋此等內容:(a)以通常形式和主部件打包到一起卻並非後者一部分,且(b)僅為和主部件一起使作品可用或實現某些已有公開實現源碼的介面。“主部件”在這裡指可執行作品運行依賴的作業系統(如果存在)的必要部件(內核、視窗系統等),生成該作品的編譯器,或運行所需的目的碼解譯器。\par
目的碼形式的作品中“相應的源碼”指所有修改作品及生成、安裝、運行(對可執行作品而言)目的碼所需的源碼,包括控制上述行為的腳本。可是,其中不包括系統庫、通用工具、未修改直接用於支援上述行為卻不是該作品一部分的通常可得的自由軟體。例如,相應的源碼包含配合作品原始檔案的介面定義,以及共用庫和作品專門依賴的動態連結副程式的源碼。這裡的依賴體現為頻密的資料交換或者該副程式和作品其他部分的控制流切換。\par
相應的源碼不必包含那些用戶可以通過源碼其他部分自動生成的內容。\par
源碼形式作品的相應源碼即其本身。
\subsection{二、基本許可}
本協定的一切授權都是對本程式的版權而言的,並且在所述條件都滿足時不可撤銷。本協議明確批准你不受限制地運行本程式的未修改版本。受保護作品的運行輸出,僅當其內容構成一個受保護作品時,才會為本協議所約束。如版權法所賦予,本協定承認你正當使用或與之等價的權利。\par
只要你獲得的許可仍有效,你可以製作、運行和傳播那些你並不轉發的受保護作品。只要你遵守本協議中關於轉發你不佔有版權的材料的條款,你可以向他人轉發,僅僅以求對方為你做定制或向你提供運行這些作品的工具。那些為你製作或運行這些受保護作品的人,應該在你的指引和控制下,謹代表你工作,即禁止他們在雙方關係之外製作任何你提供的受版權保護材料的副本。\par
僅當滿足後文所述條件時,其他各種情況下的轉發才是允許的。不允許再授權行為,而第十條的存在使再授權變得沒有必要。
\subsection{三、保護用戶的合法權益免受反破解法限制}
在任何滿足1996年12月20日通過的WIPO版權條約第11章要求的法律,或類似的禁止或限制技術手段破解的法律下,受保護作品不應該視為有效技術手段的一部分。\par
當你轉發一個受保護作品時,你將失去任何通過法律途徑限制技術手段破解的權力,乃至於通過行使本協議所予權利實現的破解。你即已表明無心通過限制用戶操作或修改受保護作品來確保你或協力廠商關於禁止技術手段破解的法定權利。
\subsection{四、轉發完整副本}
你可以通過任何媒介發佈你接收到的本程式的完整源碼副本,但要做到:為每一個副本醒目而恰當地發佈版權;完整地保留關於本協議及按第七條加入的非許可性條款;完整地保留免責聲明;給接收者附上一份本協議的副本。\par
你可以免費或收費轉發,也可以選擇提供技術支援或品質擔保以換取收入。
\subsection{五、轉發修改過的源碼版本}
你可以以源碼形式轉發基於本程式的作品或修改的內容,除滿足第四條外還需要滿足以下幾點要求:\par
a)該作品必須帶有醒目的修改聲明及相應的日期。\par
b)該作品必須帶有醒目的聲明,指出其在本協議及任何符合第七條的附加條件下發佈。這個要求修正了第四條關於“完整保留”的內容。\par
c)你必須按照本協定將該作品整體向想要獲得許可的人授權,本協議及符合第七條的附加條款就此適用於整個作品,即其每一部分,不管如何建包。本協定不允許以其他形式授權該作品,但如果你收到別的許可則另當別論。\par
d)如果該作品有互動式使用者介面,則其必須顯示適當的法律聲明。然而,當本程式有互動式使用者介面卻不顯示適當的法律聲明時,你的作品也不必。\par
一個在存儲或分發媒介上的受保護作品和其他分離的單體作品的聯合作品,在既不是該受保護作品的自然擴展,也不以構築更大的程式為目的,並且自身及其產生的版權並非用於限制單體作品給予聯合作品使用者的訪問及其他合法權利時,稱為“聚合體”。在聚合作品中包含受保護作品並不會使本協定影響聚合作品的其他部分。
\subsection{六、以非源碼形式轉發}
你可以如第四條和第五條所述那樣以目的碼形式轉發受保護作品,同時在本協定規範下以如下方式之一轉發機器可讀的對應源碼:\par
a)目的碼通過實體產品(涵蓋某種實體分發媒介)轉發時,通過常用於軟體交換的耐用型實體媒介隨同轉發相應的源碼。\par
b)目的碼通過實體產品(涵蓋某種實體分發媒介)轉發時,伴以具有至少三年且與售後服務等長有效期的書面承諾,給予目的碼的持有者:(1)包含產品全部軟體的相應源碼的常用於軟體交換的耐用型實體媒介,且收費不超過其合理的轉發成本;或者(2)通過網路免費獲得相應源碼的途徑。\par
c)單獨轉發目的碼時,伴以提供源碼的書面承諾。本選項僅在你收到目的碼及b項形式的承諾的情況下可選。\par
d)通過在指定地點提供目的碼獲取服務(無論是否收費)的形式轉發目的碼時,在同一地點以同樣的方式提供對等的源碼獲取服務,並不得額外收費。你不以要求接收者在複製目的碼的同時複製源碼。如果提供目的碼複製的地點為網路服務器,相應的源碼可以提供在另一個支援相同複製功能的伺服器上(由你或者協力廠商運營),不過你要在目的碼處指出相應源碼的確切路徑。不管你用什麼源碼伺服器,你有義務要確保持續可用以滿足這些要求。\par
e)通過點對點傳輸轉發目的碼時,告知其他節點目的碼和源碼在何處以d項形式向大眾免費提供。\par
“面向使用者的產品”指(1)“消費品”,即個人、家庭或日常用途的個人有形財產;或者(2)面向社會團體設計或銷售,卻落入居家之物。在判斷一款產品是否消費品時,爭議案例的判斷將向利於擴大保護靠攏。就特定使用者接收到特定產品而言,“正常使用”指對此類產品的典型或一般使用,不管該用戶的身份,該使用者對該產品的實際用法,以及該產品的預期用法。無論產品是否實質上具有商業上的,工業上的,及非面向消費者的用法,它都視為消費品,除非以上用法代表了它唯一的重要使用模式。\par
“安裝資訊”對面向使用者的產品而言,指基於修改過的源碼安裝運行該產品中的受保護作品的修改版所需的方法、流程、認證碼及其他資訊。這些資訊必須足以保證修改過的目的碼不會僅僅因為被修改過而不能繼續工作。\par
如果你根據本條在,或隨,或針對一款面向使用者的產品,以目的碼形式轉發某作品,且轉發體現於該產品的所有權和使用權永久或者在一定時期內轉讓予接收者的過程(無論其有何特點),根據本條進行的源碼轉發必須伴有安裝資訊。不過,如果你和協力廠商都沒有保留在該產品上安裝修改後的目的碼的能力(如作品安裝在ROM上),這項要求不成立。   要求提供安裝資訊並不要求為修改或安裝的作品,以及其載體產品繼續提供技術支援、品質擔保和升級。當修改本身對網路運行有實質上的負面影響,或違背了網路通信協定和規則時,可以拒絕其聯網。\par
根據本條發佈的源碼及安裝資訊,必須以公共的檔案格式(並且存在可用的空開源碼的處理工具)存在,同時不得對解壓、閱讀和複製設置任何密碼。
\subsection{七、附加條款}
“附加許可”用於補充本協議,以允許一些例外情況。合乎適用法律的對整個程式適用的附加許可,應該被視為本協定的內容。如果附加許可作用於程式的某部分,則該部分受此附加許可約束,而其他部分不受其影響。\par
當你轉發本程式時,你可以選擇性刪除副本或其部分的附加條款。(附加條款可以寫明在某些情況下要求你修改時刪除該條款。)在你擁有或可授予恰當版權許可的受保護作品中,你可以在你添加的材料上附加許可。\par
儘管已存在本協議的其他條款,對你添加到受保護作品的材料,你可以(如果你獲得該材料版權持有人的授權)以如下條款補充本協議:\par
a)表示不提供品質擔保或有超出十五、十六條的責任。\par
b)要求在此材料中或在適當的法律聲明中保留特定的合理法律聲明或創作印記。\par
c)禁止誤傳材料的起源,或要求合理標示修改以別於原版。\par
d)限制以宣傳為目的使用該材料的作者或授權人的名號。\par
e)降低約束以便賦予在商標法下使用商品名、商品標識及服務標識。\par
f)要求任何轉發該材料(或其修改版)並對接收者提供契約性責任許諾的人,保證這種許諾不會給作者或授權人帶來連帶責任。\par
此外的非許可性附加條款都被視作第十條所說的“進一步的限制”。如果你接收到的程式或其部分,聲稱受本協定約束,卻補充了這種進一步的限制條款,你可以去掉它們。如果某授權合約包含進一步的限制條款,但允許通過本協議再授權或轉發,你可以通過本協議再授權或轉發加入了受前協議管理的材料,不過要同時移除上述條款。\par
如果你根據本條向受保護作品添加了調控,你必須在相關的原始檔案中加入對應的聲明,或者指出哪裡可以找到它們。\par
附加條款,不管是許可性的還是非許可性的,可以以獨立的書面協議出現,也可以聲明為例外情況,兩種做法都可以實現上述要求。
\subsection{八、終止授權}
除非在本協議明確授權下,你不得傳播或修改受保護作品。其他任何傳播或修改受保護作品的企圖都是無效的,並將自動中止你通過本協議獲得的權利(包括第十一條第3段中提到的專利授權)。\par
然而,當你不再違反本協議時,你從特定版權持有人處獲得的授權恢復:(1)暫時恢復,直到版權持有人明確終止;(2)永久恢復,如果版權持有人沒能在60天內以合理的方式指出你的侵權行為。\par
再者,如果你第一次收到了特定版權持有人關於你違反本協定(對任意作品)的通告,且在收到通告後30天內改正,那你可以繼續享此有授權。\par
當你享有的權利如本條所述被中止時,已經從你那根據本協定獲得授權的他方的權利不會因此中止。在你的權利恢復之前,你沒有資格憑第十條獲得同一材料的授權。
\subsection{九、持有副本無需接受協議}
你不必為接收或運行本程式而接受本協定。類似的,僅僅因點對點傳輸接收到副本引發的對受保護作品的輔助性傳播,並不要求接受本協議。但是,除本協議外沒有什麼可以授權你傳播或修改任何受保護作品。如果你不接受本協議,這些行為就侵犯了版權。因此,一旦修改和傳播一個受保護作品,就表明你接受本協議。
\subsection{十、對下游接收者的自動授權}
每當你轉發一個受保護作品,其接收者自動獲得來自初始授權人的授權,依照本協議可以運行、修改和傳播此作。你沒有要求協力廠商遵守該協議的義務。\par
“實體事務”指轉移一個組織的控制權或全部資產、或拆分或合併組織的事務。如果實體事務導致一個受保護作品的傳播,則事務中各收到作品副本方,都有獲得前利益相關者享有或可以如前段所述提供的對該作品的任何授權,以及從前利益相關者處獲得並擁有相應的源碼的權利,如果前利益相關者享有或可以通過合理的努力獲得此源碼。\par
你不可以對本協議所授權利的行使施以進一步的限制。例如,你不可以索要授權費或版稅,或就行使本協議所授權利徵收其他費用;你也不能發起訴訟(包括交互訴訟和反訴),宣稱製作、使用、零售、批發、引進本程式或其部分的行為侵犯了任何專利。
\subsection{十一、專利}
“貢獻人”指通過本協定對本程式或其派生作品進行使用認證的版權持有人。授權作品成為貢獻人的“貢獻者版”。\par
貢獻人的“實質專利權限”指其擁有或掌控的,無論是已獲得的還是將獲得的全部專利權限中,可能被通過某種本協定允許的方式製作、使用或銷售其貢獻者版作品的行為侵犯的部分,不包括僅有修改其貢獻者版作品才構成侵犯的部分。“掌控”所指包括享有和本協議相一致的專利再授權的權利。\par
每位貢獻人皆其就實質專利權限,授予你一份全球有效的免版稅的非獨占專利許可,以製作、使用、零售、批發、引進,及運行、修改、傳播其貢獻者版的內容。\par
在以下三段中,“專利許可”指通過任何方式明確表達的不行使專利權(如對使用專利的明確許可和不起訴專利侵權的契約)的協議或承諾。對某方“授予”專利許可,指這種不對其行使專利權的協議或承諾。\par
如果你轉發的受保護作品已知依賴於某專利,而其相應的源碼並不是任何人都能根據本協定從網上或其他地方免費獲得,那你必須(1)以上述方式提供相應的源碼;或者(2)放棄從該程式的專利許可中獲得利益;或者(3)以某種和本協定相一致的方式將專利許可擴展到下游接收者。“已知依賴於”指你實際上知道若沒有專利許可,你在某國家轉發受保護作品的行為,或者接收者在某國家使用受保護作品的行為,會侵犯一項或多項該國認定的專利,而這些專利你有理由相信它們的有效性。\par
如果根據一項事務或安排,抑或與之相關,你轉發某受保護作品,或通過促成其轉手以實現傳播,並且該作品的接收方授予專利許可,以使指可以使用、傳播、修改或轉發該作品的特定副本,則此等專利許可將自動延伸及每一個收到該作品或其派生作品的人。\par
如果某專利在其涵蓋範圍內,不包含本協議專門賦予的一項或多項權利,禁止行使它們或以不行使它們為前提,則該專利是“歧視性”的。如果你和軟體發佈行業的協力廠商有合作,合作要求你就轉發受保護作品的情況向其付費,並授予作品接收方歧視性專利,而且該專利(a)與你轉發的副本(或在此基礎上製作的副本)有關,或針對包含該受保護作品的產品或聯合作品,你不得轉發本程式,除非參加此項合作或取得該專利早於2007年3月28日。\par
本協定的任何部分不應被解釋成在排斥或限制任何暗含的授權,或者其他在適用法律下對抗侵權的措施。
\subsection{十二、不得犧牲他人的自由}
即便你面臨與本協議條款衝突的條件(來自於法庭要求、協議或其他),那也不能成為你違背本協議的理由。倘若你不能在轉發受保護作品時同時滿足本協定和其他檔的要求,你就不能轉發本程式。例如,當你同意了某些要求你就再轉發問題向你的轉發物件收取版稅的條款時,唯一能同時滿足它和本協議要求的做法便是不轉發本程式。
\subsection{十三、和GNU Affero通用公共授權合約一起使用}
儘管已存在本協議的一些條款,你可以將任何受保護作品與以GNU Affero通用公共授權合約管理的作品關聯或組合成一個聯合作品,並轉發。本協定對其中的受保護作品部分仍然有效,但GNU Affero通用公共授權合約第十三條的關於網路交互的特別要求適用于整個聯合作品。
\subsection{十四、本協議的修訂版}
自由軟體聯盟可能會不定時發佈GNU通用公共授權合約的修訂版或新版。新版將秉承當前版本的精神,但對問題或事項的描述細節不盡相同。\par
每一版都會有不同的版本號,如果本程式指定其使用的GNU通用公共授權合約的版本“或任何更新的版本”,你可以選擇遵守該版本或者任何更新的版本的條款。如果本程式沒有指定協定版本,你可以選用自由軟體聯盟發佈的任意版本的GNU通用公共授權合約。\par
如果本程式指定代理來決定將來那個GNU通用公共授權合約版本適用,則該代理的公開聲明將指導你選擇協議版本。\par
新的版本可能會給予你額外或不同的許可。但是,任何作者或版權持有人的義務,不會因為你選擇新的版本而增加。
\subsection{十五、不提供品質擔保}
本程式在適用法律範圍內不提供品質擔保。除非另作書面聲明,版權持有人及其他程式提供者“概”不提供任何顯式或隱式的品質擔保,品質擔保所指包括而不僅限於有經濟價值和適合特定用途的保證。全部風險,如程式的品質和性能問題,皆由你承擔。若程式出現缺陷,你將承擔所有必要的修復和更正服務的費用。
\subsection{十六、責任範圍}
除非適用法律或書面協議要求,任何版權持有人或本程式按本協定可能存在的協力廠商修改和再發佈者,都不對你的損失負有責任,包括由於使用或者不能使用本程式造成的任何一般的、特殊的、偶發的或重大的損失(包括而不僅限於資料丟失、資料失真、你或協力廠商的後續損失、其他程式無法與本程式協同運作),即使那些人聲稱會對此負責。
\subsection{十七、第十五條和第十六條的解釋}
如果上述免責聲明和責任範圍聲明不為地方法律所支持,上訴法庭應採用與之最接近的關於放棄本程式相關民事責任的地方法律,除非本程式附帶收費的品質擔保或責任許諾。
\section{【附錄:如何將上述條款應用到你的新程式】}
如果你開發了一個新程式,並希望它能最大限度地為公眾所使用,最好的辦法是將其作為自由軟體,以使每個人都能在本協定約束下對其再發佈及修改。
為此,請在附上以下聲明。最安全的做法是將其附在每份源碼的開頭,以便於最有效地傳遞免責資訊。同時,每個檔至少包含一處“版權”聲明和一個協定全文的連結。
\begin{verbatim}
<用一行來標明程式名及其作用>
版權所有(C)<年份> <作者姓名>
本程式為自由軟體,在自由軟體聯盟發佈的GNU通用公共授權合約的約束下,
你可以對其進行再發佈及修改。協議版本為第三版或(隨你)更新的版本。
我們希望發佈的這款程式有用,但不保證,甚至不保證它有經濟價值和適合特定用途。
詳情參見GNU通用公共授權合約。
你理當已收到一份GNU通用公共授權合約的副本,如果沒有,請查閱<http://www.gnu.org/licenses/>
\end{verbatim}\par
同時提供你的電子郵寄地址或傳統的郵件聯繫方式。\par
如果該程式是互動式的,讓它在交互模式下輸出類似下面的一段聲明:
\begin{verbatim}
<程式名> 第69版,版權所有(C)<年份> <作者姓名>
本程式從未提供品質擔保,輸入'show w'可查看詳情。這是款自由軟體,歡迎你在滿足一定條件後對其再發佈,輸入'show c'可查看詳情。
\end{verbatim}\par
例子中的命令'show w'和'show c'應用於顯示GNU通用公共授權合約相應的部分。當然你也可以因地制宜地選用別的方式,對圖形介面程式可以用“關於”功能表。\par
如果你之上存在雇主(你是碼農)或校方,你還應當讓他們在必要時為此程式簽署放棄版權聲明。詳情參見<\url{http://www.gnu.org/licenses/}>。\par
本GNU通用公共授權合約不允許把你的程式併入私有程式。如果你的程式是某種庫,且你想允許它被私有程式連結而使之更有用,請使用GNU較寬鬆通用公共授權合約。決定前請先查閱<\url{http://www.gnu.org/philosophy/why-not-lgpl.html}>。 
\chapter{署名-非商业性使用-相同方式共享 3.0中国大陆}
本作品(定义如下)的提供是以适用“知识共享公共许可协议”( 简称“CCPL”或 “许可”)\footnote{来源:【知识共享许可协议法律文本】\url{https://creativecommons.org/licenses/by-nc-sa/3.0/cn/legalcode}(最后访问于2019年6月23日10:47:30)}为前提的。本作品受《中华人民共和国著作权法》以及其他可适用法律的保护。对本作品的使用不得超越本许可协议授权的范围。\par
如您行使本许可授予的使用本作品的权利,就表明您接受并同意遵守本许可协议的所有条款。鉴于本许可为合同,在您接受这些条款和规定的前提下,许可人授予您本许可所包括的权利。
\section{第一条 定义} 
\begin{enumerate}
	\item 本作品:指根据本许可协议提供的以任何方式和形式(包括以数字形式)表达之文学、艺术和科学领域的作品,例如:书籍、手册等文字作品;讲课、演讲、讲道及其他同类性质的作品;戏剧或音乐戏剧作品;曲艺作品;舞蹈作品及哑剧作品;配词或不配词的音乐作品;电影作品和以类似摄制电影的方法创作的作品;素描、绘画、书法、建筑、雕塑、雕刻或版画等作品;摄影作品以及以类似摄影的方法创作的作品;杂技艺术作品;实用艺术作品;与地理、地形、建筑或科学有关的插图、地图、设计图、草图及立体的造型作品;以及法律、行政法规规定的其他文学艺术作品。为本许可协议之目的,本协议有关“本作品”的规定适用于表演、录音制品及广播电视节目。 
	\item 原始作者:就文学或艺术作品而言,指创作本作品的自然人或依法视为本作品作者的法人或其他组织。为本许可之目的,下述情形下的自然人、法人或其他组织适用本许可有关“原始作者”的规定:(1)就表演而言,指演员、歌唱家、音乐家、舞蹈家和其他表演、演唱、演说、朗诵、演奏、表现或者以其它方式表演文学、艺术作品或民间文学艺术的人员;(2)就录音制品而言,指首次将表演的声音或其他声音录制下来的自然人、法人或其他组织;(3)就广播电视节目而言,指传播广播电视节目的组织;(4)作者身份不明的,指行使作品著作权(除署名权外)的作品原件所有人(比如出版社)。
	\item 演绎作品:指基于本作品,或基于本作品与其他已存在的作品而创作的作品,例如翻译、改编、编曲或对文学、艺术和科学作品的其他变更,包括以摄制电影的方法对作品的改编,或其他任何对本作品进行改造、转换、或改编后的形式,包含任何可确认为源自原始作品的修改形式。在本许可定义之下构成汇编作品的作品不视为演绎作品。为避免疑义,并为本许可之目的,当演绎对象为音乐作品时,将其依时间序列关系与动态影像配合一致而形成的结果,视为演绎作品。
	\item 汇编作品:指由于对内容的选择和编排具有独创性而构成智力创作的文学、艺术或科学作品的集合,其中本作品以完整且未经修改的形式和另外一部或多部作品组成集合的整体,而各组成作品本身是分开且独立的,例如百科全书、文选、数据汇编作品,以及本条第1项所列作品之外的作品或者标的。在本许可定义之下构成汇编作品的作品不视为演绎作品(定义如上)。
	\item 许可人:指根据本许可提供本作品的自然人、法人或者其他组织。
	\item 您:指以前就本作品没有违反过本许可协议、或曾违反过协议但已获得许可人明示同意、依据本许可行使权利的自然人、法人或者其他组织。
	\item 授权要素:是指许可人所选择的、并标示在本许可文本标题中的下列基本属性:署名、非商业性使用、相同方式共享。
	\item 发行:指以出售或者其他权利移转方式向公众提供本作品或者演绎作品的原件或者复制件。
	\item 公开传播:指公开朗诵本作品以及以任何方式或程序,包括以有线、无线的方式或通过信息网络公开传播本作品的公开朗诵;或向公众提供本作品,使公众可以在自己选定的地点获得本作品;或以任何方式或程序公开表演本作品或向公众传播本作品的表演,包括通过信息网络传播本作品的表演;或以任何方式,包括符号、声音或图像,广播或转播本作品。上述定义包括相关法律规定的“展览”“表演”“放映”“广播”或通过信息网络向公众传播作品等传播方式。
	\item 复制:指以印刷、复印、拓印、录音、录像、翻录、翻拍等方式制作本作品的复制件。
	\item 人身权:指相关法律赋予作者对本作品所享有的发表权、署名权、修改权以及保护作品完整权。
\end{enumerate}
\section{第二条 合理使用}
本许可无意削减、限制或约束您基于《中华人民共和国著作权法》或其他相关法律有关著作权保护的限制或例外的规定对本作品的合理使用。
\section{第三条 授权}
根据本许可的条款和条件,许可人在此授予您全球性、免版税、非独占并且在本作品的著作权存续期间内均有效的许可,就本作品行使以下权利:
\begin{enumerate}
	\item 复制本作品或将本作品收入一个或多个汇编作品中,以及复制汇编作品中收录的本作品;
	\item 创作和复制演绎作品,但是任何演绎作品,包括任何形式的翻译作品,均需以合理方式清楚地标示、区分或以其他方法表明原始作品已经被修改或变更。例如,翻译作品可以标明“原作品已由英文翻译为西班牙文”,改编作品可以标明“原作品已作修改”;
	\item 发行、公开传播本作品(包括汇编作品中收录的本作品); 
	\item 发行、公开传播演绎作品。
\end{enumerate}
以上权利可在任何现有的或者以后出现的并为可适用的法律认可的媒体和形式上行使。上述权利包括为在其他媒体和形式上行使权利而必须进行技术性修改的权利。许可人在此保留所有未明示授予的权利,包括但不限于第四条第5项所规定的权利。
\section{第四条 限制}
第三条的授权须受以下规定的限制: 
\begin{enumerate}
	\item 您在发行或公开传播本作品时,必须遵守本许可协议。在您发行或公开传播的本作品的每一份复制件中,您必须附上一份本许可协议的复制件或本许可协议的网址(Uniform Resource Identifier)。您不得就本作品提出或增加任何条款,从而限制本许可协议或者限制获得本作品的第三方行使本许可协议所赋予的权利。您不得对本作品进行再许可。您必须在您发行或公开传播的每份作品复制件中完整保留所有与本许可协议及免责条款相关的声明。 在发行或公开传播本作品时,您不得对本作品施加任何技术措施,从而限制从您处获得本作品的第三方行使本许可协议授予的权利。本项(第四条第1项)规定同样适用于收录在汇编作品中的本作品,但并不要求汇编作品中除本作品外的其他作品受本许可协议的约束。在创作汇编作品时,若接到任一许可人的通知,您必须按照其要求,在可行范围内删除汇编作品中根据本协议第四条第4项的要求所作的有关原始作者的身份及其他有关原始作品相关信息的标注。在创作演绎作品时,若接到任一许可人的通知,您必须根据其要求,在可行范围内删除演绎作品中根据第四条第4项的要求所作的有关原始作者的身份及其他有关原始作品的相关信息的标注。
	\item 您必须以下述许可条款发行或公开传播演绎作品:(1)本许可协议;(2)与本许可协议具有相同授权要素的后续版本;或者(3)与本许可协议具有相同授权要素的其他司法管辖区的知识共享许可协议或其后续版本(例如:署名-非商业性使用-相同方式共享 3.0 美国)(以上三类协议统称为“可适用的协议”)。在您发行或公开传播的每件演绎作品的复制件中,您必须附上一份“可适用的协议”的复制件或网址。您不得就演绎作品提出或增加任何条款,从而限制“可适用的协议”的规定,或者限制获得演绎作品的第三方行使“可适用的协议”所赋予的权利。在发行或公开传播包含本作品的演绎作品时,您必须在本作品的每一份复制件中完整地保留所有与“可适用的协议”及免责条款相关的声明。在发行或公开传播演绎作品时,您不得对演绎作品施加任何技术措施,从而限制从您处获得演绎作品的第三方行使“可适用的协议”所赋予的权利。本项(第四条第2项)规定同样适用于收录在汇编作品中的演绎作品,但并不要求汇编作品中除基于本作品而创作的演绎作品之外的其他作品受“可适用的协议”的约束。 
	\item 您不得以任何形式行使本协议第三条授予您的权利去谋取或获得商业利益或私人金钱报酬。若交换过程中未涉及任何商业利益或私人金钱报酬,通过数字文件共享方式或其他方式用本作品去交换其他受著作权保护的作品,将不被视为谋取或获得商业利益或私人金钱报酬。
	\item 在发行或公开传播本作品、任何演绎作品或汇编作品时,除非有依据第四条第1项之要求,否则您必须完整保留所有关于本作品的著作权声明,并以适于所使用的媒介或方法的形式提供下述信息:(1)在原始作者的姓名(或笔名)已被提供的情况下,给出该姓名或笔名,或者在原始作者或许可人以许可人的著作权声明或其他合理的方式,指定可以在作品上署名的他方当事人姓名的情况下,指明该他方当事人的名称(“署名人”);(2)在本作品标题已被提供的情况下,给出本作品的标题;(3)在合理可行的范围内,标明许可人指定需与本作品同时出现的网址,除非该网址没有涉及到本作品的著作权声明或者关于本作品的许可信息;(4)若为演绎作品,则依第三条第2项之要求,必须注明演绎作品中使用的本作品的作者姓名和作品名称(例如,“某作者作品的法语译本”,或“基于某作者作品的电影剧本”)。本项(第四条第4项)要求的对作者姓名和作品名称的指明可采取任何合理方式,但在演绎作品或汇编作品中,如果已经指明了演绎作品的所有作者或汇编作品中所有内含作品的作者,那么对本作品名称和作者姓名的指明须同时出现在任何其他作者姓名出现的地方,并至少与对其他作者的指明一样显著。为避免疑义,本条有关标示作者姓名和作品名称之规定,仅适用于前述署名的用途;除非您事先另行取得原始作者、许可人或署名人的书面同意,否则您不得以明示或者默示的方式主张或暗示,您本人或您对作品的使用与原始作者、许可人或署名人有关联或者已获得上述人士的赞助或者支持。
	\item 为避免疑义,针对不同司法管辖区的著作权许可体系作出如下约定:
	\subitem 权利不能放弃的强制许可体系。在那些许可人不能放弃通过任何法定的或强制的许可方案收取许可使用费的权利的司法管辖区,许可人保留因您行使本许可协议授予的权利而向您收取许可使用费的专有权;
	\subitem 权利可以放弃的强制许可体系。在那些许可人可以放弃通过任何法定的或强制的许可方案收取许可使用费的权利的司法管辖区,许可人放弃因您行使本许可协议授予的权利而向您收取许可使用费的专有权;但若您行使本许可协议授予的权利时未遵守本许可协议第四条第3项有关非商业性使用的规定,则许可人保留向您收取本作品许可使用费的权利;
	\subitem 自愿许可体系。在实行著作权自愿许可的司法管辖区,若您行使本许可协议授予的权利时未遵守本许可协议第四条第3项有关非商业性使用的规定,则许可人保留向您收取本作品许可使用费的权利,许可人可以自行或者通过所参加的著作权集体管理组织向您收取本作品的许可使用费。
	\item 除非其他法律法规另有规定,您在复制、发行或者公开表演本作品,或者复制、发行或者公开表演作为任何演绎作品或汇编作品一部分的本作品时,不得歪曲、损害或者以其他方式损害本作品,导致原始作者的名誉或者荣誉受损。
\end{enumerate}
\section{第五条 声明、保证和免责}
除非本许可的当事人相互以书面的方式做出相反约定,且在相关法律所允许的最大范围内,否则许可人按其现状提供本作品,对本作品不作任何明示或者默示、依照法律或者其他规定的陈述或担保,包括但是不限于任何有关可否商业性使用、是否符合特定的目的、不具有潜在的或者其他缺陷、准确性或者不存在不论能否被发现的错误的担保。有些司法管辖区不允许排除前述默示保证,因此这些排除性规定并不一定适用于您。
\section{第六条 责任限制}
除非属于相关法律所要求的范围,许可人在任何情况下都不对您因本许可或因使用本作品而产生的任何直接损失、间接损失或惩罚性赔偿负责,即使许可人已被告知发生此类损害的可能性。 
\section{第七条 许可终止}
\begin{enumerate}
	\item 在您违反本许可协议任何条款时,本许可及其所授予的权利将自动终止。然而,根据本许可从您处获取演绎作品或汇编作品的自然人、法人或者其他组织,如果他们仍完全遵守相关条款,则对他们的许可不会随之终止。即使本许可被终止,第一条、第二条、第五条、第六条、第七条以及第八条仍然有效。
	\item 在上述条款及条件的前提下,此处授予的许可在法定著作权保护期限内有效。即便如此,许可人保留依其他许可条款发行本作品及在任何时候停止发行本作品的权利;但是,许可人的上述权利不能被用于撤销本许可或任何其他在本许可条款下授予的或必须授予的许可,除本条第1项指明的终止外,本许可将保持其完全效力。
\end{enumerate}
\section{第八条 其他事项}
\begin{enumerate}
	\item 当您发行、公开传播本作品或其汇编作品时,许可人给获得作品的第三方提供本作品的许可,其条款和条件与您所获得的许可相同。
	\item 当您发行或公开传播演绎作品时,许可人给获得作品的第三方提供本作品的许可,其条款和条件与您所获得的许可相同。
	\item 如因相关法律,本许可的某一条款无效或不能履行,本许可其余条款的有效性和可履行性不受影响。如本许可的当事人未采取进一步措施,此类无效条款应在必要范围内进行最低限度的修改以使其有效和可履行。
	\item 除非当事人书面同意并签字放弃某条款和允许某违约行为,本许可的任何条款或规定都不应被视为已被放弃,或被视为允许此违约行为。
	\item 本许可构成相关当事人与本授权作品相关的全部协议。除已在此处确认的之外,并不存在任何与本作品相关的谅解备忘录、协议或声明。许可人不受您提出的任何附加规定的约束。未经许可人和您双方书面同意,本许可不得更改。
\end{enumerate}
\section{“知识共享”(Creative Commons)声明}
“知识共享”不是本许可协议的一方,对本作品不作任何相关保证。“知识共享”不对您或任何其他方的任何损失负责,包括但不限于与本许可相关的直接损失和间接损失。虽然有上述两点,但如果“知识共享”已明确标识自己为许可人,它将承担许可人的全部权利和义务。\par
除用于向公众表明本作品是依照知识共享公共许可协议(CCPL)授权以外,如未经“知识共享”事先书面同意,任何一方均不得使用“知识共享”(Creative Commons)商标和其他相关商标及标识。任何被允许的使用都必须符合“知识共享”的现行商标使用准则。该准则已在其网站上发布或可应要求随时提供。为避免疑义,本段关于商标的限制性规定不构成本许可之一部分。\par
您可以通过\url{https://creativecommons.org/}(最后连接于2019年6月23日10:57:40)联系“知识共享”